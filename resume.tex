%-------------------------
% Resume in Latex
% Author : Juan Pinto
% Based off of: https://github.com/jakegut/resume
% License : MIT
%------------------------

\documentclass[letterpaper,11pt]{article}

\usepackage{latexsym}
\usepackage[empty]{fullpage}
\usepackage{titlesec}
\usepackage{marvosym}
\usepackage[usenames,dvipsnames]{color}
\usepackage{verbatim}
\usepackage{enumitem}
\usepackage[hidelinks]{hyperref}
\usepackage{fancyhdr}
\usepackage[english]{babel}
\usepackage{tabularx}
\input{glyphtounicode}


%----------FONT OPTIONS----------
% sans-serif
% \usepackage[sfdefault]{FiraSans}
% \usepackage[sfdefault]{roboto}
% \usepackage[sfdefault]{noto-sans}
% \usepackage[default]{sourcesanspro}

% serif
% \usepackage{CormorantGaramond}
% \usepackage{charter}


\pagestyle{fancy}
\fancyhf{} % clear all header and footer fields
\fancyfoot{}
\renewcommand{\headrulewidth}{0pt}
\renewcommand{\footrulewidth}{0pt}

% Adjust margins
\addtolength{\oddsidemargin}{-0.5in}
\addtolength{\evensidemargin}{-0.5in}
\addtolength{\textwidth}{1in}
\addtolength{\topmargin}{-.5in}
\addtolength{\textheight}{1.0in}

\urlstyle{same}

\raggedbottom
\raggedright
\setlength{\tabcolsep}{0in}

% Sections formatting
\titleformat{\section}{
  \vspace{-4pt}\scshape\raggedright\large
}{}{0em}{}[\color{black}\titlerule \vspace{-5pt}]

% Ensure that generate pdf is machine readable/ATS parsable
\pdfgentounicode=1

%-------------------------
% Custom commands
\newcommand{\resumeItem}[1]{
  \item\small{
    {#1 \vspace{-2pt}}
  }
}

\newcommand{\resumeSubheading}[4]{
  \vspace{-2pt}\item
    \begin{tabular*}{0.97\textwidth}[t]{l@{\extracolsep{\fill}}r}
      \textbf{#1} & #2 \\
      \textit{\small#3} & \textit{\small #4} \\
    \end{tabular*}\vspace{-7pt}
}

\newcommand{\resumeSubSubheading}[2]{
    \item
    \begin{tabular*}{0.97\textwidth}{l@{\extracolsep{\fill}}r}
      \textit{\small#1} & \textit{\small #2} \\
    \end{tabular*}\vspace{-7pt}
}

\newcommand{\resumeProjectHeading}[2]{
    \item
    \begin{tabular*}{0.97\textwidth}{l@{\extracolsep{\fill}}r}
      \small#1 & #2 \\
    \end{tabular*}\vspace{-7pt}
}

\newcommand{\resumeSubItem}[1]{\resumeItem{#1}\vspace{-4pt}}

\renewcommand\labelitemii{$\vcenter{\hbox{\tiny$\bullet$}}$}

\newcommand{\resumeSubHeadingListStart}{\begin{itemize}[leftmargin=0.15in, label={}]}
\newcommand{\resumeSubHeadingListEnd}{\end{itemize}}
\newcommand{\resumeItemListStart}{\begin{itemize}}
\newcommand{\resumeItemListEnd}{\end{itemize}\vspace{-5pt}}

%-------------------------------------------
%%%%%%  RESUME STARTS HERE  %%%%%%%%%%%%%%%%%%%%%%%%%%%%


\begin{document}

%----------HEADING----------
% \begin{tabular*}{\textwidth}{l@{\extracolsep{\fill}}r}
%   \textbf{\href{http://sourabhbajaj.com/}{\Large Sourabh Bajaj}} & Email : \href{mailto:sourabh@sourabhbajaj.com}{sourabh@sourabhbajaj.com}\\
%   \href{http://sourabhbajaj.com/}{http://www.sourabhbajaj.com} & Mobile : +1-123-456-7890 \\
% \end{tabular*}

\begin{center}
    \textbf{\Huge \scshape Juan D. Pinto} \\ \vspace{1pt}
    % \small 801-228-7184 $|$
    \href{mailto:juan@jdpinto.com}{juan@jdpinto.com} $|$
    % \href{mailto:jdpinto2@illinois.edu}{jdpinto2@illinois.edu} $|$
    \href{https://jdpinto.com}{jdpinto.com} $|$
    \href{https://github.com/juandpinto}{github.com/juandpinto}
    % \href{https://www.linkedin.com/in/juandpinto/}{\underline{linkedin.com/in/jdpinto}} $|$
\end{center}


%-----------EXPERIENCE-----------
\section{Experience}
  \resumeSubHeadingListStart

    \resumeSubheading
    {\href{https://www.google.org}{Google.org} + \href{https://aiinstitutes.org}{AI Institutes Virtual Organization}}{Remote}
    %   {[Upcoming] AIVO Summer Graduate Fellow}{[Upcoming] May 2025 -- Aug 2025}
      {[Upcoming] Google.org AI4Ed Research Fellow}{[Upcoming] May 2025 -- Aug 2025}
      \resumeItemListStart
        \resumeItem{Will engage in collaborative projects across NSF AI in Education Institutes}
      \resumeItemListEnd

    \resumeSubheading
    {\href{https://invite.illinois.edu}{NSF AI Institute for Inclusive Intelligent Technologies for Education (INVITE)}}{Urbana, IL}
    {Learner Modeling Graduate Research Assistant}{Aug 2023 -- Present}
      \resumeItemListStart
        % \resumeItem{Participated as a core member of the \emph{student modeling} team}
        \resumeItem{Developed predictive models of student skills and behaviors for real-time adaptive learning}
        \resumeItem{Led data cleaning and analysis on datasets with tens- to hundreds-of-thousands of student actions}
        % \resumeItem{Conducted cleaning and analyses on large-scale educational datasets from various digital platforms}
      \resumeItemListEnd

    \resumeSubheading
      {\href{https://heds-lab.com}{Human-Centered Educational Data Science Lab (HEDS)}}{Urbana, IL}
      {Graduate Research Assistant}{Sept 2020 -- Present}
      \resumeItemListStart
        % \resumeItem{Led and contributed to research projects in the fields of \emph{learning analytics} and \emph{educational data mining}}
        \resumeItem{Developed predictive and inferential models of student behaviors, emphasizing explainable AI}
        \resumeItem{Investigated CS students' coding patterns using epistemic network analysis, LLMs, and various ML approaches}
        \resumeItem{Contributed to 13 peer-reviewed publications (6 as lead author) in venues related to \emph{educational data mining}}
    \resumeItemListEnd

    \resumeSubheading
    {\href{https://www.ets.org}{ETS Research Institute}}{Princeton, NJ}
    {Ida Lawrence Research Intern}{June 2024 -- July 2024}
      \resumeItemListStart
        \resumeItem{Developed small heuristic classification models (for ensembling) that detect student reading disengagement}
        \resumeItem{Validated models indirectly (unlabeled data) using response accuracy, on-task behavior, and book preferences}
      \resumeItemListEnd

  \resumeSubHeadingListEnd


%-----------PROJECTS-----------
\section{Projects}
    \resumeSubHeadingListStart

      \resumeProjectHeading
          {\textbf{Interpretable Neural Network for Learner Behavior Detection} $|$ \emph{Python, PyTorch}}{}%{Apr 2023 -- Present}
          \resumeItemListStart
            \resumeItem{Developed a convolutional neural network for detecting rare gaming-the-system behavior among learners}
            \resumeItem{Emphasized interpretable-by-design approach via custom loss function and novel thresholding mechanism}
            \resumeItem{Demonstrated that the model provides fully faithful explanations utilizing 100\% of its inference-time parameters}
            % \resumeItem{Evaluated the model's performance and explainability against human expert-identified patterns}
            \resumeItem{Achieved 90\% explanation intelligibility among human users}
            % \resumeItem{Developed a convolutional neural network to detect learner behaviors}
            % \resumeItem{Designed model constraints to simplify inference and align learned patterns with human intuition}
            % \resumeItem{Achieved comparable accuracy with previously published models}
            % \resumeItem{Evaluated faithfulness and intelligibility of the model's explanations using human subjects}
          \resumeItemListEnd

      \resumeProjectHeading
          {\textbf{Evaluating LLMs for Debugging Strategy Classification} $|$ \emph{Python, Scikit-learn}}{}
          \resumeItemListStart
            \resumeItem{Developed pipeline for systematic LLM prompting across different dimensions, such as \emph{chain-of-thought}, \emph{zero-} vs. \emph{few-shot}, \emph{single-} vs. \emph{multi-label}, \emph{reasoning}, and \emph{fine-tuned}}
            \resumeItem{Trained+tuned various ML models to compare against LLM results for classifying students' debugging strategies}
            \resumeItem{Improved annotation efficiency and minority class detection in student code}
          \resumeItemListEnd

      \resumeProjectHeading
          {\textbf{Weight-Based Modeling for Student Performance Prediction} $|$ \emph{Python, Scikit-learn, PyTorch, TensorFlow}}{}
          \resumeItemListStart
            % \resumeItem{Developed weighting schemes to predict student performance using programming traces}
            \resumeItem{Engineered complex weighted features to predict student performance on future coding problems}
            % \resumeItem{Designed similarity metrics based on code, problem prompts, and struggling patterns}
            \resumeItem{Showed that \emph{source code} and \emph{struggling pattern} similarity, along with \emph{problem order}, improved prediction accuracy} % by ___%
            \resumeItem{Demonstrated that logistic regression with weighting schemes matched SOTA model performance}
            \resumeItem{Won 2nd place in the \href{https://sites.google.com/ncsu.edu/csedm-dc-2021/home}{2022 Educational Data Mining in CS Data Challenge}}
          \resumeItemListEnd

      \resumeProjectHeading
          {\textbf{Modeling Student Performance Using Measures of Persistence} $|$ \emph{Python, Scikit-learn}}{}%{Nov 2020 -- Apr 2021}
          \resumeItemListStart
            \resumeItem{Tested multiple linear, tree-based, and ensemble models to predict student quiz performance using homework data}
            \resumeItem{Conducted careful feature engineering based on previously studied elements of student persistence}
            \resumeItem{Analyzed the role of features and their interactions in-depth using SHAP values from random forest model}
          \resumeItemListEnd

    %   \resumeProjectHeading
    %       {\textbf{Epistemic Network Analysis of CS Students' Debugging Behavior} $|$ \emph{R, Python}}{}%{Jan 2024 -- May 2024}
    %       \resumeItemListStart
    %         \resumeItem{Analyzed debugging behaviors in novice programmers using Epistemic Network Analysis (ENA)}
    %         \resumeItem{Identified key constituents of the debugging process based on expert interpretation of student behaviors}
    %         \resumeItem{Compared debugging strategies between students with different prior programming experience}
    %         \resumeItem{Investigated how debugging behaviors evolved over time as students gained experience in an intro CS course}
    %       \resumeItemListEnd


    %   \resumeProjectHeading
    %       {\textbf{Micro-Models of Student Engagement} $|$ \emph{Python, Scikit-learn}}{}
    %       \resumeItemListStart
    %         \resumeItem{Developed simple interpretable models (for ensembling) that detect student disengagement in a reading app}
    %         \resumeItem{Analyzed response times and task delays as disengagement signals}
    %         \resumeItem{Validated models using accuracy, on-task behavior, and book preferences}
    %       \resumeItemListEnd

    \resumeSubHeadingListEnd


%-----------EDUCATION-----------
\section{Education}
  \resumeSubHeadingListStart
    \resumeSubheading
      {University of Illinois Urbana-Champaign}{Urbana, IL}
      {Ph.D. in Educational Data Science}{Aug 2025}
    \resumeSubheading
      {University of Michigan}{Ann Arbor, MI}
      {M.A. in Design and Technologies for Learning}{July 2020}
    % \resumeSubheading
    %   {University of Texas at Austin}{Austin, TX}
    %   {M.A. in Middle Eastern Languages and Cultures}{Aug 2016 -- May 2018}
    \resumeSubheading
      {Brigham Young University}{Provo, UT}
      {B.A. in Ancient Near Eastern Studies}{May 2016}
  \resumeSubHeadingListEnd


%
%-----------TECHNICAL SKILLS-----------
\section{Skills}
 \begin{itemize}[leftmargin=0.15in, label={}]
    \small{\item{
     \textbf{Data Analysis \& Visualization}{: Python (NumPy, Pandas, Matplotlib, Seaborn), R, SQL} \\
    %  \textbf{Data Visualization}{: Matplotlib, Seaborn} \\
     \textbf{Machine Learning \& AI}{: Scikit-learn, PyTorch, Tensorflow, Keras} \\
    %  \textbf{Web Programming}{: HTML, CSS, Javascript, Hugo, Jekyll} \\
    % \textbf{Other}{: Git, \LaTeX, Adobe Creative Cloud (Photoshop, Illustrator)}
    }}
 \end{itemize}


%-------------------------------------------
\end{document}
